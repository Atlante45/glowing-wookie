\section{Raspberry Pi}
\subsection{La classe \texttt{Protocol}}

La classe \texttt{Protocol} regroupe l'ensemble de fonctions permettant à l'utilisateur d'envoyer des commandes et de recevoir les réponses. Cette classe contient l'état du protocol, contenant un ensemble de valeurs telles que le nombre de pins disponibles sur la carte esclave, ou le compteur de réponses permettant de détecter une erreur de communication ou un éventuel \emph{reset} de la carte. L'avantage d'une implémentation orientée objet réside ici dans la simplification des méthodes accessibles à l'utilisateur.

\subsection{Synchrone/Asynchrone}
Les méthodes proposées permettent une utilisation synchrone ou non du protocole.
Il est important de préciser que les méthodes destinées à une utilisation synchrone sont bloquantes : elles attendent la réponse à la commande envoyée, et transmettent les résultats, automatiquement extraits de la réponse, à l'utilisateur via des références passées en argument; ou retournent un code d'erreur en cas de problème. Si la communication est intérrompue, un délais d'expiration permet un débloquage.

Pour ce qui est des méthodes fonctionnant de manière asynchrone, la méthode \texttt{receiveCommand()} peut être utilisée pour extraire les données de la réponses mais c'est à l'utilisateur de les analysées et les exploiter.


\subsection{Communication sur le port série}
La communication sur le port série s'effectue grâce à la bibliothèque \texttt{serialib}. Celle-ci permet d'envoyer et de recevoir des données brutes sur le port série.
 
