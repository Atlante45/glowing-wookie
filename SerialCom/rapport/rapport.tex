\documentclass[a4paper,10pt]{article}

\usepackage[french]{babel}
\usepackage[utf8]{inputenc}
\usepackage[T1]{fontenc}
\usepackage{lmodern}

\usepackage[bottom=3cm,left=3cm]{geometry}
\usepackage[urlcolor=blue,colorlinks=true]{hyperref}

\usepackage{graphics}
\usepackage{graphicx}

\usepackage{listings}
\lstset{frame=tb,
  language=C,
  aboveskip=3mm,
  belowskip=3mm,
  showstringspaces=false,
  columns=flexible,
  basicstyle={\small\ttfamily},
  numbers=none,
  numberstyle=\tiny\color{gray},
  keywordstyle=\color{blue},
  commentstyle=\color{red},
  stringstyle=\color{mauve},
  breaklines=true,
  breakatwhitespace=true
  tabsize=4
}

\begin{document}

\vspace*{\stretch{5}}
\noindent{}
\rule{\textwidth}{1pt}
\begin{flushright}
  {\Huge Projet Système Embarqué}

~

  {\Large Protocole de communication série}

~

  {\large  Maxime \textsc{Bellier}, Clement \textsc{Brisset}, Cedric \textsc{Jolys}, Thibaud \textsc{Lambert} }
\end{flushright}
\rule{\textwidth}{1pt}
\thispagestyle{empty}
\vspace*{\stretch{5}}
\begin{flushright} \today \end{flushright}

\newpage

\tableofcontents

\newpage

\section{Introduction}
Dans ce projet, nous nous intéressons à l'implémentation d'un protocole série.
Ce protocole a pour but de permettre la communication entre une carte maître et une carte esclave.
Le but est de proposer à l'utilisateur une série de fonction permettant la lecture et l'écriture de pin se trouvant sur la carte esclave.
Le protocole doit être implémenté de façon générique afin de gérer une carte esclave avec un nombre de pins pouvant varier de 0 à 255.


\section{Partie commune}
Afin de factoriser au maximum le code, un ensemble de modules ont été regroupé dans un dossier \texttt{common} de manière à être utilisé à la fois côté maître (Raspberry Pi) et côté esclave (Atmega 8).
Il est important de rappeler que l'implémentation est faite en C++ sur la Raspberry Pi, et en C sur l'Atmega 8. C'est pourquoi les fichiers communs ont été codés en C.

\subsection{Protocole}
Le fichier \texttt{protocol\_command.h} rassemble l'ensemble des informations spécifiques au protocole. 
Parmis celles-ci, se trouvent l'offset, la taille en octets, et la taille et en bits des différents champs.
L'ensemble du code utilise la convention de nommage suivante afin de différencier la taille en octets et en bits:
\begin{itemize}
\item un identifiant finissant par \texttt{LENGTH} désigne la taille en octets.
\item un identifiant finissant par \texttt{SIZE} désigne la taille en bits.
\end{itemize}

Un énumération permet de décrire les différentes valeurs des types.
Deux tableaux contiennent la taille en octets et en bits des types indexés par l'énumération.


Le fichier \texttt{protocol\_util.c} contient une fonction permettant la création complète d'un paquet à partir du header et du payload.
Cette fonction se charge d'allouer le buffer, de calculer la taille complète du paquet et d'appliquer la checksum.

\subsection{Lecture et écriture binaire}
La totalité des manipulations bit à bit du projet sont effectuées grâce aux fonctions du module \texttt{bits.h}. La fonction \texttt{binary\_set()} permet de modifier la valeur d'un bit au sein d'une chaîne. La \texttt{binary\_get()} permet quant à elle d'extraire une série de bits au sein d'un octet.

Ces fonctions sont alors utilisée par \texttt{binary\_read()} et \texttt{binary\_write()} qui sont capables de lire et écrire une série de bits donnée, représentée par une valeur numérique, depuis et vers une chaîne de caractères

Grâce à ces fonctions, il est alors possible de lire et écrire une série de bits n'importe où dans une chaîne de caractères.


\subsection{Masque}
Dans le but de faciliter la manipulation de masques, une structure et un petit set de fonctions sont fournis dans les fichiers \texttt{mask.h} et \texttt{mask.c}. En particulier, des fonctions permettent de lire et écrire directement des masques depuis une chaîne de bits.

Voici tout d'abords la structure \texttt{mask\_t} qui en elle même est très simple :
\lstset{language=C}
\begin{lstlisting}
// mask.h
typedef
struct mask {
  int *values;
  unsigned int nb_values;
  unsigned int value_size;
} mask_t;
\end{lstlisting}

L'attribut \texttt{value\_size} spécifie le nombre de bits occupé par chaque valeur. Le fait de mettre dans celui-ci une valeur supérieure à 1 permet d'étendre l'utilisation à celle de tableaux dont la taille en bits de chaque cellule n'est pas un multiple de 8.





\section{Atmega 8}
Le code dans sur l'Atmega 8 se divise en trois principaux modules.
Le module \texttt{communication} permet l'envoie et la recption de chaine de caractére.
Le module \texttt{pinAction} rassemble les fonctions permettant la lecture et l'écriture sur les différentes pins.
Le module \texttt{protocol} se charge de parser les chaines reçues, d'effectuer les actions correspondantes et de renvoyer la reponse.

\subsection{Communication sur le port série}
Afin d'initialiser les communications sur la carte, la fonction \texttt{init\_com()} doit être appelé.
La premiére étape est d'activer les pins TX et RX dans le registre \texttt{UCSRB}.
Ensuite, dans le registre \texttt{UCSRC} on définit la taille d'un caractére à 8 bits.
Finalement, la baudrate est indiqué dans les registres \texttt{UBRRH} et \texttt{UBRRL}.

L'envoie et la reception passe par le registre \texttt{UDR}.
En effet le registre \texttt{UDR} permet d'accéder aux \texttt{TXB} et \texttt{RXB} qui correspondent respectivement au buffer d'écriture et au buffer de lecture.
Quand une écriture est effectué sur le registre \texttt{UDR} les données sont envoyées dans TXB.
Quand une lecture est faites sur le registre \texttt{UDR}, le registre RXB est lue à la place.

En consultant les bits \texttt{UDRE} du registre \texttt{UCSRA} ont peut déterminer si le registre \texttt{UDR} est pret pour l'envoie de données.
De même, en consultant les bits \texttt{RXC} du registre \texttt{UCSRA} ont peut savoir si des données ont été reçue.

\subsection{Lecture et écriture sur les pins}
Pour spécifier si les pins sont en lectures ou en écriture, il faut passer par les registres \texttt{DDRD}, \texttt{DDRB} et \texttt{DDRC}.
Si le bit est à 0 alors la pin est en mode lecture, sinon la pin est en mode écriture.


\subsection{Protocole}


\section{Raspberry Pi}
\subsection{La classe \texttt{Protocol}}
Par soucis d'encapsulation et de facilité d'utilisation, la totalité des mécanismes du protocole sont codés dans cette classe.
Celle ci sauvegarde les états inhérent au protocole et à sa validité comme le nombre de pins disponibles après que cela est été demandé par l'utilisateur.
Ou bien le reply ID, ce qui permet à la classe protocole de détecter un reset et le signaler à l'utilisateur.

Sont aussi implémentés toutes les commandes fournies par le protocole dans des méthodes publiques afin qu'une fois une instance crée, l'utilisateur soit capable d'envoyer n'importe quelles commandes de façon simplifiée en spécifiant seulement ce qu'il souhaite dans la payload.

\subsection{Synchrone/Asynchrone}
Les méthodes fournies permettent une utilisation synchrone ou non du protocole.
Il est important de préciser que les méthodes destinées à une utilisation synchrone sont bloquante, car elles attendent la réponse à la comande envoyer afin de pouvoir charcher le résultat dans les pointeurs et référence adéquates passés en argument ainsi que renvoyer un code d'erreur si jamais il y a eu un problème. En cas de reset non désiré de l'\texttt{Atmega 8} celles le detectent et renvoie le code d'erreur approprié.

Pour ce qui est des méthodes fonctionnant de manière asynchrone, il est nécéssaire d'utiliser une méthode supplémentaire appelée \texttt{recieveCommand()} qui chargera les données du packet reçue dans les références passées en argument.
C'est alors à l'utilisateur de gérer les données reçues bien que \texttt{recieveCommand()} fasse déjà une partie du traitement.


\subsection{Communication sur le port série}
Pour la communication sur le port série, c'est la librairie nommée \texttt{seriallib} qui est utilisée. Celle-ci fourni un set de fonction parmettant d'envoyer et de recevoir des packet une fois que ceux ci sont construit.
 
Ayant rencontré des problèmes avec l'\texttt{Atmega 8}, il a alors été décidé d'ajouter une option disponible via la commande \texttt{make DEBUG=1} qui remmplace les fonctions de communications de l'\texttt{Atmega 8} et du \texttt{Raspberry Pi} par des fonctions communicant via des fichiers de type fifo afin de pouvoir faciliter le debug et la vérification du fontionnement du protocole.

\section{Tests sur ordinateur}

Afin de pouvoir tester le code sur un ordinateur sans utiliser une connexion série, une implémentation faussaire des modules d'accès au port série permet d'utiliser deux FIFOs pour communiquer. Cette implémentation est disponible via une option du makefile, \texttt{make DEBUG=1}. 

Sur l'\emph{AtMega}, les mocks ont également pour but de remplacer toutes les parties du code propres à avr et ainsi de permettre la compilation sur une architecture x86.

Grâce a ce système, il est ainsi possible de tester que les paquets sont bien générés et bien analysés.



\section{Conclusion}

La finalité première de ce projet était d'améliorer la latence de la communication série. Toutefois, nous avons concentré notre démarche autours de l'implémentation du protocole. En effet, l'écrasement du bootloader de la carte \emph{AtMega} nous a empêché de poursuivre au niveau matériel, que ce soit au niveau de l'optimisation de la taille du binaire ou de l'amélioration de la latence.


\end{document}
