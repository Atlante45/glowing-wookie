\documentclass[a4paper,10pt]{article}

\usepackage[french]{babel}
\usepackage[utf8]{inputenc}
\usepackage[T1]{fontenc}
\usepackage{lmodern}

\usepackage[bottom=3cm,left=3cm]{geometry}
\usepackage[urlcolor=blue,colorlinks=true]{hyperref}

\usepackage{graphics}
\usepackage{graphicx}

\usepackage{listings}
\lstset{frame=tb,
  language=C,
  aboveskip=3mm,
  belowskip=3mm,
  showstringspaces=false,
  columns=flexible,
  basicstyle={\small\ttfamily},
  numbers=none,
  numberstyle=\tiny\color{gray},
  keywordstyle=\color{blue},
  commentstyle=\color{red},
  stringstyle=\color{mauve},
  breaklines=true,
  breakatwhitespace=true
  tabsize=4
}

\begin{document}

\vspace*{\stretch{5}}
\noindent{}
\rule{\textwidth}{1pt}
\begin{flushright}
  {\Huge Projet Système Embarqué}

~

  {\Large Protocole de communication série}

~

  {\large  Maxime \textsc{Bellier}, Clement \textsc{Brisset}, Cedric \textsc{Jolys}, Thibaud \textsc{Lambert} }
\end{flushright}
\rule{\textwidth}{1pt}
\thispagestyle{empty}
\vspace*{\stretch{5}}
\begin{flushright} \today \end{flushright}

\newpage

\tableofcontents

\newpage

\section{Introduction}
Dans ce projet, nous nous intéressons à l'implémentation d'un protocole série.
Ce protocole a pour but de permettre la communication entre une carte maître et une carte esclave.
L'objectif est de proposer à l'utilisateur une série de fonctions permettant la lecture et l'écriture de pins se trouvant sur la carte esclave.
Le protocole doit être implémenté de façon générique afin de gérer une carte esclave avec un nombre de pins pouvant varier de 0 à 255.


\section{Partie commune}
Afin de factoriser au maximum le code, dans un dossier \texttt{common} un ensemble de modules ont été regroupé de manière à être utilisé à la fois côté maître (Raspberry Pi) et côté esclave (Atmega 8).

\subsection{Protocole}
Le fichier \texttt{protocol\_command.h} rassemble l'ensemble des informations spécifique au protocole. 
On y trouve l'offset, la taille en octets et en bits des différents champs.
La convention suivante a été choisi afin de différencier la taille en octets et en bits:
\begin{itemize}
\item un define finissant par \texttt{LENGTH} désigne la taille en octets.
\item un define finissant par \texttt{SIZE} désigne la taille en bits.
\end{itemize}

Un énumération permet de décrire les différents valeurs des types.
Deux tableaux contiennent la taille en octets et en bits des types indexés par l'énumération.


Le fichier \texttt{protocol\_util.c} contient une fonction permettant la création complète d'un paquet à partir du header et du payload.
Cette fonction se charge d'allouer le buffer, de calculer la taille complète du paquet et la checksum.

\subsection{Lecture et écriture binaire}

\subsection{Masque}

\section{Atmega 8}
Le code dans sur l'Atmega 8 se divise en trois principaux modules.
Le module \texttt{communication} permet l'envoie et la recption de chaine de caractére.
Le module \texttt{pinAction} rassemble les fonctions permettant la lecture et l'écriture sur les différentes pins.
Le module \texttt{protocol} se charge de parser les chaines reçues, d'effectuer les actions correspondantes et de renvoyer la reponse.

\subsection{Communication sur le port série}
Afin d'initialiser les communications sur la carte, la fonction \texttt{init\_com()} doit être appelé.
La premiére étape est d'activer les pins TX et RX dans le registre \texttt{UCSRB}.
Ensuite, dans le registre \texttt{UCSRC} on définit la taille d'un caractére à 8 bits.
Finalement, la baudrate est indiqué dans les registres \texttt{UBRRH} et \texttt{UBRRL}.

L'envoie et la reception passe par le registre \texttt{UDR}.
En effet le registre \texttt{UDR} permet d'accéder aux \texttt{TXB} et \texttt{RXB} qui correspondent respectivement au buffer d'écriture et au buffer de lecture.
Quand une écriture est effectué sur le registre \texttt{UDR} les données sont envoyées dans TXB.
Quand une lecture est faites sur le registre \texttt{UDR}, le registre RXB est lue à la place.

En consultant les bits \texttt{UDRE} du registre \texttt{UCSRA} ont peut déterminer si le registre \texttt{UDR} est pret pour l'envoie de données.
De même, en consultant les bits \texttt{RXC} du registre \texttt{UCSRA} ont peut savoir si des données ont été reçue.

\subsection{Lecture et écriture sur les pins}
Pour spécifier si les pins sont en lectures ou en écriture, il faut passer par les registres \texttt{DDRD}, \texttt{DDRB} et \texttt{DDRC}.
Si le bit est à 0 alors la pin est en mode lecture, sinon la pin est en mode écriture.


\subsection{Protocole}


\section{Raspberry Pi}
\subsection{La classe \texttt{Protocol}}

La classe \texttt{Protocol} regroupe l'ensemble de fonctions permettant à l'utilisateur d'envoyer des commandes et de recevoir les réponses. Cette classe contient l'état du protocol, contenant un ensemble de valeurs telles que le nombre de pins disponibles sur la carte esclave, ou le compteur de réponses permettant de détecter une erreur de communication ou un éventuel \emph{reset} de la carte. L'avantage d'une implémentation orientée objet réside ici dans la simplification des méthodes accessibles à l'utilisateur.

\subsection{Synchrone/Asynchrone}
Les méthodes proposées permettent une utilisation synchrone ou non du protocole.
Il est important de préciser que les méthodes destinées à une utilisation synchrone sont bloquantes : elles attendent la réponse à la commande envoyée, et transmettent les résultats, automatiquement extraits de la réponse, à l'utilisateur via des références passées en argument; ou retournent un code d'erreur en cas de problème. Si la communication est intérrompue, un délais d'expiration permet un débloquage.

Pour ce qui est des méthodes fonctionnant de manière asynchrone, la méthode \texttt{receiveCommand()} peut être utilisée pour extraire les données de la réponses mais c'est à l'utilisateur de les analysées et les exploiter.


\subsection{Communication sur le port série}
La communication sur le port série s'effectue grâce à la bibliothèque \texttt{serialib}. Celle-ci permet d'envoyer et de recevoir des données brutes sur le port série.
 

\section{Tests sur ordinateur}

Afin de pouvoir tester le code sur un ordinateur sans utiliser une connexion série, une implémentation faussaire des modules d'accès au port série permet d'utiliser deux FIFOs pour communiquer. Cette implémentation est disponible via une option du makefile, \texttt{make DEBUG=1}. 

Sur l'\emph{AtMega}, les mocks ont également pour but de remplacer toutes les parties du code propres à avr et ainsi de permettre la compilation sur une architecture x86.

Grâce a ce système, il est ainsi possible de tester que les paquets sont bien générés et bien analysés.



\section{Conclusion}

La finalité première de ce projet était d'améliorer la latence de la communication série. Toutefois, nous avons concentré notre démarche autours de l'implémentation du protocole. En effet, l'écrasement du bootloader de la carte \emph{AtMega} nous a empêché de poursuivre au niveau matériel, que ce soit au niveau de l'optimisation de la taille du binaire ou de l'amélioration de la latence.


\end{document}
