\section{Partie commune}
Afin de factoriser au maximum le code, un ensemble de modules ont été regroupé dans un dossier \texttt{common} de manière à être utilisé à la fois côté maître (Raspberry Pi) et côté esclave (Atmega 8).
Il est important de rappeler que l'implémentation est faite en C++ sur la Raspberry Pi, et en C sur l'Atmega 8. C'est pourquoi les fichiers communs ont été codés en C.

\subsection{Protocole}
Le fichier \texttt{protocol\_command.h} rassemble l'ensemble des informations spécifiques au protocole. 
Parmis celles-ci, se trouvent l'offset, la taille en octets, et la taille et en bits des différents champs.
L'ensemble du code utilise la convention de nommage suivante afin de différencier la taille en octets et en bits:
\begin{itemize}
\item un identifiant finissant par \texttt{LENGTH} désigne la taille en octets.
\item un identifiant finissant par \texttt{SIZE} désigne la taille en bits.
\end{itemize}

Un énumération permet de décrire les différentes valeurs des types.
Deux tableaux contiennent la taille en octets et en bits des types indexés par l'énumération.


Le fichier \texttt{protocol\_util.c} contient une fonction permettant la création complète d'un paquet à partir du header et du payload.
Cette fonction se charge d'allouer le buffer, de calculer la taille complète du paquet et d'appliquer la checksum.

\subsection{Lecture et écriture binaire}
La totalité des manipulations bit à bit du projet sont effectuées grâce aux fonctions du module \texttt{bits.h}. La fonction \texttt{binary\_set()} permet de modifier la valeur d'un bit au sein d'une chaîne. La \texttt{binary\_get()} permet quant à elle d'extraire une série de bits au sein d'un octet.

Ces fonctions sont alors utilisée par \texttt{binary\_read()} et \texttt{binary\_write()} qui sont capables de lire et écrire une série de bits donnée, représentée par une valeur numérique, depuis et vers une chaîne de caractères

Grâce à ces fonctions, il est alors possible de lire et écrire une série de bits n'importe où dans une chaîne de caractères.


\subsection{Masque}
Dans le but de faciliter la manipulation de masques, une structure et un petit set de fonctions sont fournis dans les fichiers \texttt{mask.h} et \texttt{mask.c}. En particulier, des fonctions permettent de lire et écrire directement des masques depuis une chaîne de bits.

Voici tout d'abords la structure \texttt{mask\_t} qui en elle même est très simple :
\lstset{language=C}
\begin{lstlisting}
// mask.h
typedef
struct mask {
  int *values;
  unsigned int nb_values;
  unsigned int value_size;
} mask_t;
\end{lstlisting}

L'attribut \texttt{value\_size} spécifie le nombre de bits occupé par chaque valeur. Le fait de mettre dans celui-ci une valeur supérieure à 1 permet d'étendre l'utilisation à celle de tableaux dont la taille en bits de chaque cellule n'est pas un multiple de 8.




