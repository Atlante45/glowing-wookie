\section{Partie commune}
Afin de factoriser au maximum le code, dans un dossier \texttt{common} un ensemble de modules ont été regroupé de manière à être utilisé à la fois côté maître (Raspberry Pi) et côté esclave (Atmega 8).

\subsection{Protocole}
Le fichier \texttt{protocol\_command.h} rassemble l'ensemble des informations spécifique au protocole. 
On y trouve l'offset, la taille en octets et en bits des différents champs.
La convention suivante a été choisi afin de différencier la taille en octets et en bits:
\begin{itemize}
\item un define finissant par \texttt{LENGTH} désigne la taille en octets.
\item un define finissant par \texttt{SIZE} désigne la taille en bits.
\end{itemize}

Un énumération permet de décrire les différents valeurs des types.
Deux tableaux contiennent la taille en octets et en bits des types indexés par l'énumération.


Le fichier \texttt{protocol\_util.c} contient une fonction permettant la création complète d'un paquet à partir du header et du payload.
Cette fonction se charge d'allouer le buffer, de calculer la taille complète du paquet et la checksum.

\subsection{Lecture et écriture binaire}

\subsection{Masque}
