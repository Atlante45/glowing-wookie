\section{Atmega 8}
Le code dans sur l'Atmega 8 se divise en trois principaux modules.
Le module \texttt{communication} permet l'envoie et la réception de chaine de caractères.
Le module \texttt{pinAction} rassemble les fonctions permettant la lecture et l'écriture sur les différentes pins.
Le module \texttt{protocol} se charge d'analyser les chaînes reçues, d'effectuer les actions correspondantes et de renvoyer la réponse.

\subsection{Communication sur le port série}
Afin d'initialiser les communications sur la carte, la fonction \texttt{init\_com()} doit être appelée.
La premiére étape est d'activer les pins TX et RX dans le registre \texttt{UCSRB}.
Ensuite, la taille d'un caractére à 8 bits est définie dans le registre \texttt{UCSRC}.
Finalement, la baudrate est indiquée dans les registres \texttt{UBRRH} et \texttt{UBRRL}.

L'envoie et la réception passe par le registre \texttt{UDR}.
En effet, le registre \texttt{UDR} permet d'accéder aux \texttt{TXB} et \texttt{RXB} qui correspondent respectivement au buffer d'écriture et au buffer de lecture.
Quand une écriture est effectuée sur le registre \texttt{UDR} les données sont envoyées dans TXB.
Quand une lecture est faite sur le registre \texttt{UDR}, le registre RXB est lu à la place.

En consultant les bits \texttt{UDRE} du registre \texttt{UCSRA}, il est possible de déterminer si le registre \texttt{UDR} est pret pour l'envoi de données.
De même, consulter les bits \texttt{RXC} du registre \texttt{UCSRA} permet de vérifier si des données ont été reçu.




\subsection{Lecture et écriture sur les pins}
Pour spécifier si les pins sont en lectures ou en écriture, il faut passer par les registres \texttt{DDRD}, \texttt{DDRB} et \texttt{DDRC}.
Si le bit est à 0 alors la pin est en mode lecture, sinon la pin est en mode écriture.
Une fois ces registres configurés, on accède aux pins à travers les registres \texttt{PINB}, \texttt{PINC} et \texttt{PIND}.


\subsection{Protocole}

Une fonction principale \texttt{parseProtocol} permet de faire le traitement complet d'une commande.
On récupère le header et la taille du paquet.
Connaissant la taille du paquet, il est possible de récupérer le payload et finalement la checksum.
Ensuite, il s'agit d'analyser le header afin de déterminer le type de la commande puis d'appeller la fonction spécifique au traitement de cette commande.

Une structure state a été implementée afin de sauvegarder l'état courant de la carte, en particulier l'état des pins et l'identifiant de réponse.

