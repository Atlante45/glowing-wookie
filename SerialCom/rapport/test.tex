\section{Tests sur ordinateur}

Afin de pouvoir tester le code sur un ordinateur sans utiliser une connexion série, une implémentation faussaire des modules d'accès au port série permet d'utiliser deux FIFOs pour communiquer. Cette implémentation est disponible via une option du makefile, \texttt{make DEBUG=1}. 

Sur l'\emph{AtMega}, les mocks ont également pour but de remplacer toutes les parties du code propres à avr et ainsi de permettre la compilation sur une architecture x86.

Grâce a ce système, il est ainsi possible de tester que les paquets sont bien générés et bien analysés.
